\begin{center}
\thispagestyle{empty}
\vspace{2cm}
\LARGE{\textbf{ABSTRACT}}\\[1.0cm]
\end{center}
\thispagestyle{empty}
\Large{}We present and discuss several novel applications of machine learning (ML) for the physical layer, by interpreting a communications system as an autoencoder and analyse the role of convolutional neural networks and deep neural networks on raw IQ samples for modulation classification.
\section*{Blind Pre-processor Modulation Classifier}
We tend to build a deep neural network to help classify modulation used to transmit the digital data before processing stage by using supervised learning and tend to show that they  outperform traditional classification techniques based on expert features. 
\vspace{-1cm}
\Large{}\paragraph{}
\section*{Modelling Communication Channels as Autoencoders}
\Large{}\paragraph{} We address the problem of learning an efficient and adaptive physical layer encoding to communicate binary information over an impaired channel. In contrast to traditional work, we treat the problem an unsupervised machine learning problem .We tend to reconstruct the wireless channel model as an end-to-end model that seeks to jointly optimize transmitter and receiver components in a single process.

\vspace{1cm}
\textbf{Keywords:} {Supervised Learning, Autoencoder, Unsupervised Learning, Modulation, Physical Layer }