
\chapter{Introduction}
\paragraph{}Radio communication/wireless communication is all about finding efficient ways to transfer information over a channel which can be as simple as corruption with a Gaussian noise to more complex channels exhibiting features like multi-path fading, impulsive noise and so on. 
\paragraph{}Theoretically we often know the upper or lower bounds on various quantities like achievable capacity and information density vs bit error rates for given modulation schemes, bandwidths and signal to noise ratios etc. In practice we have been able to much closer to them but this usually involves a complex and expensive hardware support and DSP software tuning which may not be feasible in everyday practice.
\paragraph{}By taking approach of unsupervised and supervised learning we sought to optimize the reconstruction of a communication channel or classify modulation schemes at the preprocessor level. 

\section{Potential of ML for the physical layer}
\paragraph{} ML algorithms could provide improvement over existing physical layer algorithms. Most common models in wireless systems involve linear, stationary mathematical models and have Gaussian statistics. Real-life systems however have a lot of imperfections and non-linearities. For this reason deep learning bases systems can be optimized for a particular hardware configuration. Moreover most of the communication system models involve individual optimization of the various blocks involved in the transmission of the signal like source, channel, modulation, encoding, equalization. Although this system has given us very efficient models but it not entirely clear that making such individual components efficient will lead to best end-to-end performance. Attempts to jointly optimize these components as it done in machine learning however will guarantee us the best end-to-end performance. 


\section{Project Goal \& Specification}
\paragraph{}The goal of our project to look closely at 2 applications of machine learning in communication systems. The first one involves the classification of modulation at the pre-processor level, which will help us reduce the overhead in transmission of signals which usually involves encoding the type of modulation in the transmitted signal as well. The second application involves using unsupervised learning to model an end-to-end communications systems by optimization of both the trasnmitter and receiver model as one deep neural network that can be trained as an autoencoder to minimize the reconstruction loss and show that this achieves equivalent BLER compared to the traditional methods of hamming encoding.The beauty of this approach is that it can even be applied to channel models and loss functions for which the optimal solutions are unknown.

